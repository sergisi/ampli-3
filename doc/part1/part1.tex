\begin{document}
	En aquest apartat, es detallarà les decisions que es van creure oportunes per a realitzar el patró i els requeriments
	del treball.
\subsection{Map}
%La interficie de injectors semblava un HashMap
%El problema que s'ens oferia era la decisió de clau valor
%Les claus són el que ens demanen a getObject
Es va decidir utilitzar un Map, ja que els mètodes de la mateixa interfície d'injector portaven a pensar amb un Map. En aquest, existeix
el mètode de "get" per una clau amb el mètode \texttt{getObject}, i tres mètodes per a afegir valors
amb \texttt{registerConstant, registerFactory} i \texttt{registerSingleton}.
\subsubsection{Value}
El problema que resta és el tipus de valor que es guardarà en el diccionari. No es podia guardar instàncies d'objecte, ja que podria
ser que un \texttt{Factory} necessites tornar sempre elements diferents per un camp intern, com per exemple, una classe
\texttt{Factory} que crea noves identificacions pels clients que la utilitzen.\\
\\
Descartant la possibilitat de guardar valors instanciats només quedava guardar funcions. La idea és utilitzar una
funció del tipus \texttt{() -> a} (en notació similar a Haskell) per a extreure
la idea de constant, \texttt{Singleton} i \texttt{Factory}.\\
\\
A més a més, s'ha optat per fer una altra interfície per a fer-ho, que s'ha anomenat \texttt{FunctionObject}. Realment
es podia haver utilitzat la interfície de \texttt{Factory} donada, però s'ha cregut que era més costós d'entendre que
guardava el Map, pel que es va optar per fer una interfície a part. Finalment, per a implementar la detecció
de cicles, es necessitava passar per parametre un conjunt per a controlar si un node ja s'havia visitat 
amb anterioritat.
% Si fos una instància d'objecte llavors Factory no es podira implementar perque tornariem sempre el mateix objecte
%Per tant s'ha decidit implementar-ho com una funció
%La funció és un Factory de zero parametres
\subsection{Throwables}
S'ha fet ús de l'excepció DependencyException pels casos mencionats a l'enunciat
de la pràctica. El que val més la pena mencionar és el cas que preveu crear un objecte
que forma un cicle de dependències.\\
\\
Per trobar el cicle de dependències s'utilitza un set, on s'afegeixen
les claus que s'han demanat quan es fa la crida a \texttt{getObject}. Si en un moment
donat es demana una clau que ja està al set voldrà dir que s'ha creat un cicle
de dependències, i es llençarà l'excepció.
%Es va interretar que haviem de llançar una exepció quan registravem un objecte i ja estava registrat i quan es treia un objecte que no estava registrat
\subsection{Implementacions}
Per requeriment del treball, s'han de fer les implementacions mandroses. D'aquesta forma, sigui l'ordre que sigui quan
es registrin les dependències, quan es resolgui el valor es retornarà el valor sempre que es pugui. Per a fer-ho es va
crear el mètode \texttt{getObjects}, el qual tradueix en les crides a les classes les dades. Amagar l'execució d'aquest
dins d'una funció anònima, fa que el mètode s'executi només quan es reclama la instància amb \texttt{getObject},
fent així una avaluació "mandrosa".
% Es va intentar que tots els calculs generals és fessin de forma mandrosa.
\subsubsection{registerSingelton()}
Els mètodes de constant i \texttt{Factory} són trivials donat aquest disseny, però el mètode de \texttt{Singleton} s'ha
hagut de realitzar de forma diferent. L'objectiu en el qual es volia arribar era que en el primer \texttt{getObject} es
crea per la instància, però la resta de vegades no. Per a guardar aquest estat s'ha utilitzat una classe anònima,
on existeix un objecte primerament instanciat a \texttt{null}. Quan es cridi per primer cop en el \texttt{getObject}, 
es crearà l'objecte, la resta de vegades, es retornarà el mateix.
%Es va utilitzar una clase anònima en comptes d'una lambda perque podia mantenir l'estat de diferents crides
%El calcul del singuelton només es fa quan es demana per primer cop, no quan es registre.
\subsection{Dificultats}
%TODO al no poder fer els throws dins del map es van haver de fer apart a la funció getObjects
%TODO no es va realitzar el primer cop la implementació de la detecció de cicles.
En la realització d'aquesta pràctica s'han trobat diferents punts que han fet un canvi en el disseny primerament plantejat.
El primer dels canvis que es va fer, va ser la decisió d'implementar \texttt{getObjects}. En la primera prova del disseny es
va intentar fer-ho amb \textit{streams} i un \texttt{map}, però que aquest mètode pugues llançar errors feia impossible
la implementació directe. Es podia haver amagat la implementació amb un \texttt{try-catch}, i retornant un Optional, però
llavors es requeria d'un control addicional del mateix error, pel que es va creure oportú fer-ho amb un mètode que pogués
llançar l'excepció tranquil·lament\\
\\
El segon cas, va ser causat per una mala lectura de l'enunciat, fet que va portar a no implementar la detecció del cicle de 
dependències. Donat a que l'anterior disseny era robust, ajustar-se a aquest requeriment només es va necessitar canviar
les classes de \texttt{Container}, i la classe \texttt{FunctionToObject} que compartient  entre elles. 
Abans d'aquest canvi, \texttt{FunctionToObject} no requeria de paràmetres.
no requeria 
\end{document}