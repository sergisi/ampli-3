\begin{document}
\subsection{Map}
%TODO La interficie de injectors semblava un HashMap
%TODO El problema que s'ens oferia era la decisió de clau valor
%TODO Les claus són el que ens demanen a getObject
Es va decidir utilitzar un Map, ja que els mètodes de la mateixa interfície d'injector portaven a pensar amb un Map. En aquest, existeix
el mètode de "get" per una clau amb el mètode \texttt{getObject}, i tres mètodes per a afegir valors 
amb \texttt{registerConstant, registerFactory} i \texttt{registerSingleton}. 
\subsubsection{Value}
El problema que resta és el tipus de valor que es guardarà en el diccionari. No es podia guardar instàncies d'objecte, ja que podria
ser que un \texttt{Factory} necessites tornar sempre elements diferents per un camp intern, com per exemple, una classe
\texttt{Factory} que crea noves identificacions pels clients que la utilitzen.\\
\\
Descartant la possibilitat de guardar valors instanciats només quedava guardar funcions. La idea és utilitzar una
funció del tipus \texttt{Set k -> a} (en notació similar a Haskell) per a extreure
la idea de constant, \texttt{Singleton} i \texttt{Factory}. Aquesta funció rebra per parametre un Set que contindrà totes les dependencies que s'hauran anat injectant a mesura que s'executa el programa. Un cop s'intenti inserir una dependencia que ja es troba dins del set, es llençarà una excepció.\\
\\
A més a més, s'ha optat per fer una altra interfície per a fer-ho, que s'ha anomenat \texttt{FunctionToObject}. Realment
es podia haver utilitzat la interfície de \texttt{Factory} donada, però s'ha cregut que era més costós d'entendre que 
guardava el Map, pel que es va optar per fer una interfície a part.
%TODO Si fos una instancia d'objecte llavors Factory no es podira implementar perque tornariem sempre el mateix objecte
%TODO Per tant s'ha decidit implementar-ho com una funció 
%TODO La funció és un Factory de zero parametres
\subsection{Throwables}
%TODO Es va interretar que haviem de llançar una exepció quan registravem un objecte i ja estava registrat i quan es treia un objecte que no estava registrat
\subsection{Implementacions}
Per objectiu propi, es va voler que les implementacions fossin mandroses. D'aquesta forma, sigui l'ordre que sigui quan
es registrin les dependències, quan es resolgui el valor es retornarà el valor sempre que es pugui. Per a fer-ho es va
crear el mètode \texttt{getObjects}, el qual tradueix en les crides a les classes les dades.
%TODO Es va intentar que tots els calculs generals és fessin de forma mandrosa.
\subsubsection{registerSingelton()}
Els mètodes de constant i  \texttt{Factory} són trivials donat aquest disseny, però el mètode de \texttt{Singleton} s'ha
hagut de realitzar de forma diferent. L'objectiu en el qual es volia arribar era que en el primer \texttt{getObject} es
crea per primer cop la instància, però la resta de vegades no. Per a guardar aquest estat s'ha utilitzat una classe anònima,
on existeix un objecte primerament instanciat a null. Quan es cridi per primer cop en el "get", es crearà l'objecte, la resta
de vegades, es retornarà el mateix.
%TODO Es va utilitzar una clase anònima en comptes d'una lambda perque podia mantenir l'estat de diferents crides
%TODO El calcul del singuelton només es fa quan es demana per primer cop, no quan es registre.
\subsection{Dificultats}
Es volia realitzar el \texttt{getObjects} com un \texttt{map} de \texttt{getObjects}, però el mètode llança exceptions
quan no es troba la dependència, pel que no s'ha pogut realitzar.
%TODO al no poder fer els throws dins del map es van haver de fer apart a la funció getObjects
\end{document}
